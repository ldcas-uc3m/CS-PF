\section{Introducción}
En la era de la digitalización, la eficiencia y la accesibilidad en los trámites administrativos se han convertido en elementos fundamentales para mejorar la experiencia del ciudadano. La transformación digital no solo simplifica los procesos, sino que también impulsa una interacción más directa y segura entre los usuarios y las instituciones.

\subsection{Sistema a desarrollar}
La propuesta consiste en una plataforma digital integral basada en agentes LLM (modelos de lenguaje a gran escala) que permita automatizar y optimizar la gestión de trámites y consultas administrativas. Entre sus principales características se destacan:

\begin{itemize}
    \item Interfaz Multicanal: Disponibilidad de una plataforma web y aplicaciones móviles con interfaces intuitivas para una comunicación fluida mediante chatbots y asistentes virtuales.
    \item Procesamiento del Lenguaje Natural: Empleo de tecnología avanzada para interpretar y responder consultas en lenguaje natural, adaptándose a las necesidades específicas de cada usuario.
    \item Automatización de Procesos: Integración con sistemas de gestión documental y bases de datos oficiales, permitiendo la verificación y el seguimiento en tiempo real de los trámites.
    \item Seguridad y Privacidad: Implementación de protocolos robustos de autenticación y cifrado, garantizando la protección y confidencialidad de la información.
    \item Soporte y Escalabilidad: Capacidad de escalar y combinar respuestas automatizadas con atención personalizada cuando sea necesario.
    \item Integración con Plataformas Existentes: Conexión con sistemas de información y servicios externos para facilitar la interoperabilidad y la integración de datos.
    \item Personalización y Recomendaciones: Adaptación de la plataforma a las preferencias y necesidades de cada usuario, ofreciendo sugerencias y guías personalizadas.
\end{itemize}

\subsection{Nombre de la empresa}
La empresa se llamará BuroSmart. Este nombre refleja la fusión de la burocracia con la inteligencia tecnológica, resaltando el compromiso de transformar procesos tradicionales en soluciones digitales modernas y accesibles.

\subsection{Misión}
La misión de BuroSmart es facilitar el acceso a trámites y consultas administrativas a través de soluciones tecnológicas basadas en inteligencia artificial. Se busca simplificar los procesos burocráticos, ofreciendo respuestas precisas, rápidas y seguras que mejoren la experiencia del ciudadano y reduzcan la complejidad en la gestión de trámites.

\subsection{Visión}
La visión de BuroSmart es convertirse en el referente en innovación tecnológica aplicada a la gestión administrativa. La empresa se proyecta como líder en la transformación digital del sector público y privado, creando un entorno más eficiente, accesible y transparente para todos los ciudadanos, y promoviendo una cultura digital que responda a las necesidades emergentes de la sociedad.




