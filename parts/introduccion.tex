\section{Introducción}
En la era de la digitalización, la eficiencia y la accesibilidad en los trámites administrativos se han convertido en elementos fundamentales para mejorar la experiencia del ciudadano. La transformación digital no solo simplifica los procesos, sino que también impulsa una interacción más directa y segura entre los usuarios y las instituciones.


\subsection{Nombre de la empresa}
La empresa se llamará BuroSmart. Este nombre refleja la fusión de la burocracia con la inteligencia tecnológica, resaltando el compromiso de transformar procesos tradicionales en soluciones digitales modernas y accesibles.

\subsection{Misión}
La misión de BuroSmart es facilitar el acceso a trámites y consultas administrativas a través de soluciones tecnológicas basadas en inteligencia artificial. Se busca simplificar los procesos burocráticos, ofreciendo respuestas precisas, rápidas y seguras que mejoren la experiencia del ciudadano y reduzcan la complejidad en la gestión de trámites.

\subsection{Visión}
La visión de BuroSmart es convertirse en el referente en innovación tecnológica aplicada a la gestión administrativa. La empresa se proyecta como líder en la transformación digital del sector público y privado, creando un entorno más eficiente, accesible y transparente para todos los ciudadanos, y promoviendo una cultura digital que responda a las necesidades emergentes de la sociedad.

\subsection{Sistema a desarrollar}

\subsubsection{Descripción general}
La transformación digital en la administración pública es imperativa para responder a las demandas actuales de eficiencia, transparencia y accesibilidad. La complejidad y la lentitud de los procesos burocráticos generan barreras que impiden un servicio óptimo para el ciudadano. Este sistema se propone como respuesta innovadora a estos desafíos, facilitando la interacción con las instituciones y mejorando la experiencia del usuario.

Este sistema se concibe como una plataforma digital integral destinada a transformar y simplificar los trámites y consultas administrativas. Su diseño se fundamenta en el uso de agentes basados en modelos de lenguaje a gran escala (LLM) que permiten interpretar y gestionar solicitudes en lenguaje natural, automatizando procesos que tradicionalmente han sido lentos y complejos. La plataforma integra múltiples canales de acceso, tales como aplicaciones web y móviles, y se conecta con sistemas oficiales y bases de datos, garantizando una operación en tiempo real. Además, se han considerado altos estándares de seguridad y cumplimiento normativo para proteger la información sensible de los usuarios.

El sistema se orienta a mejorar la eficiencia y la calidad de los servicios administrativos, reduciendo los tiempos de respuesta y aumentando la satisfacción del ciudadano. Asimismo, busca fomentar la transparencia y la trazabilidad de los procesos, generando informes y análisis de datos que permitan una toma de decisiones más informada y una mejora continua en la gestión pública.

\subsubsection{Funcionalidades del sistema}
Descripción de la Operatividad:

\begin{itemize}
    \item Interfaz Multicanal: Provisión de una plataforma web y aplicaciones móviles con interfaces intuitivas que permiten la interacción mediante chatbots y asistentes virtuales.
    \item Automatización y Procesamiento del Lenguaje Natural: Empleo de agentes LLM para interpretar y gestionar consultas y trámites de forma automatizada y en lenguaje natural.
    \item Gestión Documental y Notificaciones: Digitalización de documentos y envío de notificaciones electrónicas para la recepción y seguimiento de trámites.
    \item Análisis de Datos y Reporting: Generación de informes y análisis de datos para la toma de decisiones y la mejora continua de los procesos administrativos.
    \item Integración de Sistemas: Conexión con bases de datos oficiales y sistemas de gestión documental para la verificación y seguimiento de los procesos administrativos.
    \item Seguridad y Cumplimiento: Implementación de protocolos avanzados de autenticación, cifrado y normativas de protección de datos, garantizando la confidencialidad y la integridad de la información.
    \item Soporte y Escalabilidad: Ofrece la posibilidad de canalizar consultas complejas a atención humana especializada, garantizando un servicio integral y adaptable a distintos volúmenes de usuarios.
    \item Personalización y Experiencia del Usuario: Adaptación de la plataforma a las necesidades y preferencias de los usuarios, ofreciendo una experiencia personalizada y satisfactoria.
\end{itemize}

\subsubsection{Objetivos estratégicos}
El sistema persigue alcanzar varios objetivos estratégicos a corto, mediano y largo plazo.

\textbf{Objetivos a corto plazo:}
Los objetivos que se tiene para corto plazo son los siguientes:

\begin{itemize}
    \item Optimización Inmediata de Procesos: 
    \begin{itemize}
        \item Reducir tiempos de respuesta en la gestión de trámites.
        \item Automatizar tareas rutinarias para disminuir la intervención manual.
    \end{itemize}
    \item Mejora de la Experiencia del Usuario:
    \begin{itemize}
        \item Ofrecer un servicio accesible y amigable para el ciudadano.
        \item Implementar canales de comunicación direct
        \item Garantizar la disponibilidad y la confiabilidad de la plataforma.
    \end{itemize}
    \item Garantía de seguridad:
    \begin{itemize}
        \item Establecer protocolos de autenticación y cifrado robustos para proteger los datos desde el lanzamiento inicial.
    \end{itemize}
\end{itemize}

\textbf{Objetivos a mediano plazo:}
Los objetivos que se tiene para mediano plazo son los siguientes:

Transparencia y Seguimiento de Trámites:
Desarrollar herramientas que permitan a los usuarios monitorear el estado de sus solicitudes en tiempo real.
Implementar sistemas de notificaciones y alertas que informen sobre cambios en el estado de los trámites.
Integración y Conectividad con Sistemas Oficiales:
Ampliar la integración con bases de datos y sistemas de gestión documental de entidades oficiales para mejorar la veracidad y actualización de la información.
Adaptabilidad y Escalabilidad del Sistema:
Diseñar la arquitectura modular del sistema para facilitar futuras expansiones o adaptaciones a cambios normativos y tecnológicos.
Optimización de la Automatización Híbrida:
Perfeccionar la combinación entre respuestas automáticas y escalado a atención humana especializada para casos complejos.

\begin{itemize}
    \item Transparencia y Seguimiento de Trámites:
    \begin{itemize}
        \item Desarrollar herramientas que permitan a los usuarios monitorear el estado de sus solicitudes en tiempo real.
        \item Implementar sistemas de notificaciones y alertas que informen sobre cambios en el estado de los trámites.
    \end{itemize}
    \item Integración y Conectividad con Sistemas Oficiales:
    \begin{itemize}
        \item Ampliar la integración con bases de datos y sistemas de gestión documental de entidades oficiales para mejorar la veracidad y actualización de la información.
    \end{itemize}
    \item Adaptabilidad y Escalabilidad del Sistema:
    \begin{itemize}
        \item Diseñar la arquitectura modular del sistema para facilitar futuras expansiones o adaptaciones a cambios normativos y tecnológicos.
    \end{itemize}
    \item Optimización de la Automatización Híbrida:
    \begin{itemize}
        \item Perfeccionar la combinación entre respuestas automáticas y escalado a atención humana especializada para casos complejos.
        \item Implementar sistemas de aprendizaje automático para mejorar la precisión y la eficiencia de los agentes LLM.
    \end{itemize}
\end{itemize}

\textbf{Objetivos a largo plazo:}
Los objetivos que se tiene para largo plazo son los siguientes:

\begin{itemize}
    \item Liderazgo en Innovación Tecnológica:
    \begin{itemize}
        \item Posicionar la plataforma como referencia en soluciones digitales para la administración pública y privada.
        \item Continuar incorporando tecnologías emergentes para mantener el sistema a la vanguardia.
    \end{itemize}
    \item Transformación Digital Integral de la Gestión Administrativa:
    \begin{itemize}
        \item Promover una cultura digital en el ámbito burocrático, impulsando la eficiencia y la transparencia en todos los procesos administrativos.
        \item Fomentar la colaboración y la interoperabilidad entre diferentes instituciones y organismos a nivel nacional.
    \end{itemize}
    \item Sostenibilidad y Contribución Social:
    \begin{itemize}
        \item Contribuir a la reducción de desigualdades mediante el acceso equitativo a servicios administrativos.
        \item Mantener el compromiso con los Objetivos de Desarrollo Sostenible, asegurando que la innovación tecnológica se alinee con el desarrollo social y económico a largo plazo.
    \end{itemize}
\end{itemize}

\subsubsection{Elementos de innovación}
El sistema se distingue por varios aspectos innovadores que le permiten superar los modelos tradicionales:
\begin{itemize}
    \item \textbf{Uso de Agentes LLM}: La incorporación de modelos de lenguaje a gran escala permite una comprensión profunda del lenguaje natural, ofreciendo respuestas más humanas y precisas que los sistemas convencionales.
    \item \textbf{Automatización Inteligente y Híbrida}: Combina la automatización total con la posibilidad de intervención humana cuando se detectan casos complejos, lo que asegura un equilibrio entre eficiencia y calidad en el servicio.
    \item \textbf{Diseño Centrado en el Usuario}: La interfaz y la experiencia de usuario están diseñadas para ser lo más intuitivas posible, facilitando el acceso y uso del sistema para personas con distintos niveles de competencia digital.
    \item \textbf{Seguridad de Última Generación}: Implementa tecnologías avanzadas de cifrado y autenticación, estableciendo nuevos estándares en la protección de la información en el ámbito administrativo.
    \item \textbf{Integración de Sistemas y Datos en Tiempo Real}: La capacidad de conectarse con múltiples fuentes oficiales garantiza que la información sea siempre actual y verificable, mejorando la confiabilidad del sistema.
\end{itemize}

\subsubsection{Contribución a los ODS}
El proyecto se alinea estratégicamente con varios de los Objetivos de Desarrollo Sostenible, aportando valor social y medioambiental:
\begin{itemize}
    \item \textbf{ODS 9 – Industria, Innovación e Infraestructura}: Impulsa la creación de infraestructuras tecnológicas modernas e innovadoras que potencian la eficiencia y la conectividad.
    \item \textbf{ODS 10 – Reducción de las Desigualdades}: Facilita el acceso a servicios administrativos a todos los ciudadanos, reduciendo barreras y promoviendo la equidad en el acceso a la información.
    \item \textbf{ODS 16 – Paz, Justicia e Instituciones Sólidas}: Contribuye a la transparencia y la rendición de cuentas en la gestión pública, fortaleciendo las instituciones y la confianza ciudadana en el Estado.
\end{itemize}

\subsubsection{Casos de uso relevantes}
Se identifican varios escenarios que ejemplifican cómo interactuarán los usuarios con el sistema:
\begin{itemize}
    \item \textbf{Consulta de Estado de Trámite:} \\
    \textbf{Actor:} Ciudadano \\
    \textbf{Descripción:} El usuario ingresa a la plataforma para verificar el estado de un trámite, recibiendo detalles actualizados sobre su avance, posibles requerimientos adicionales y tiempo estimado de resolución.
    \item \textbf{Solicitud de Asesoramiento Personalizado:} \\
    \textbf{Actor:} Ciudadano con consultas complejas \\
    \textbf{Descripción:} En caso de una consulta que requiera interpretación o solución específica, el sistema analiza la solicitud y, si es necesario, canaliza la atención a un agente especializado que complementa la respuesta automatizada.
    \item \textbf{Actualización y Validación de Datos Administrativos:} \\
    \textbf{Actor:} Funcionario o Administrador \\
    \textbf{Descripción:} Permite a los funcionarios acceder a la plataforma para actualizar información oficial en tiempo real, validar la veracidad de los datos ingresados por los ciudadanos y asegurar la integridad de la base de datos.
    \item \textbf{Gestión de Incidencias y Retroalimentación:} \\
    \textbf{Actor:} Ciudadano y Soporte Técnico \\
    \textbf{Descripción:} Los usuarios pueden reportar incidencias o sugerencias, y el sistema registra estas entradas para que el equipo de soporte técnico pueda dar seguimiento y mejorar continuamente el servicio.
\end{itemize}

\subsubsection{Diagramas UML}
Para complementar la documentación técnica y facilitar la comprensión del sistema, se recomienda la elaboración de los siguientes diagramas UML:
\begin{itemize}
    \item \textbf{Diagrama de Estados:} \\
    \textbf{Propósito:} Representar los distintos estados por los que puede transitar un trámite (por ejemplo, "Iniciado", "En Proceso", "Completado", "Rechazado") y los eventos que provocan el cambio de estado. \\
    Comentario: Se requiere un diagrama de estados que muestre las transiciones clave de un trámite administrativo, incluyendo eventos desencadenantes y condiciones de transición.
    \item \textbf{Diagrama de Secuencia:} \\
    \textbf{Propósito:} Ilustrar la interacción dinámica entre los actores (ciudadano, funcionario, agente LLM) y los componentes del sistema durante un proceso típico, como la consulta o actualización de un trámite. \\
    Comentario: Se necesita un diagrama de secuencia que detalle el flujo de mensajes y acciones desde el inicio de la solicitud del usuario hasta la respuesta final del sistema.
    \item \textbf{Diagrama de Actividad:} \\
    \textbf{Propósito:} Visualizar el flujo de trabajo completo de un proceso administrativo, desde la recepción de la solicitud hasta su resolución final, identificando decisiones y actividades clave en el proceso. \\
    Comentario: Se requiere un diagrama de actividad que muestre las actividades secuenciales y paralelas, junto con las bifurcaciones y convergencias del proceso administrativo.
\end{itemize}
