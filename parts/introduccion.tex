\section{Introducción}
En la era de la digitalización, la eficiencia y la accesibilidad en los trámites administrativos se han convertido en elementos fundamentales para mejorar la experiencia del ciudadano. La transformación digital no solo simplifica los procesos, sino que también impulsa una interacción más directa y segura entre los usuarios y las instituciones.


\subsection{Nombre de la empresa}
La empresa se llamará BuroSmart. Este nombre refleja la fusión de la burocracia con la inteligencia tecnológica, resaltando el compromiso de transformar procesos tradicionales en soluciones digitales modernas y accesibles.

\subsection{Misión}
La misión de BuroSmart es facilitar el acceso a trámites y consultas administrativas a través de soluciones tecnológicas basadas en inteligencia artificial. Se busca simplificar los procesos burocráticos, ofreciendo respuestas precisas, rápidas y seguras que mejoren la experiencia del ciudadano y reduzcan la complejidad en la gestión de trámites.

\subsection{Visión}
La visión de BuroSmart es convertirse en el referente en innovación tecnológica aplicada a la gestión administrativa. La empresa se proyecta como líder en la transformación digital del sector público y privado, creando un entorno más eficiente, accesible y transparente para todos los ciudadanos, y promoviendo una cultura digital que responda a las necesidades emergentes de la sociedad.


\subsection{Sistema a desarrollar}

\subsubsection{Descripción general}
por que, que hace, objetivos, por que es innovador
ods
casos de uso, diagramas de despliegue, de componente
UML diagrama de estados, secuencia y actividad

\subsubsection{Funcionalidades del sistema}

\subsubsection{Objetivos estratégicos}

\subsubsection{Elementos de innovación}

\subsubsection{Contribución a los ODS}

\subsubsection{Casos de uso relevantes}

\subsubsection{Diagramas UML}

