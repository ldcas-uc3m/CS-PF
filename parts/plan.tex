\section{Plan de negocio}

\subsection{Organización de la empresa}
La empresa cuenta con una estructura matricial, la cual ofrece una gran flexibilidad y colaboración entre los distintos departamentos, y permite enfocarse en varias funciones a la vez. Es cierto que este tipo de estructura genera un doble reporte en ciertas situaciones, y puede generar conflictos entre gerentes, pero también permite a cada equipo centrarse en su función sin perder su especialización.

La estructura organizativa se divide por centros de responsabilidad, con un responsable por centro. La \figref{organigrama} muestra el organigrama de la empresa.

\drawiosvgfigure{organigrama}{Organigrama de la empresa}


\subsubsection{Centros de responsabilidad}
Los centros de responsabilidad, con sus respectivos responsables son los siguientes:
\begin{itemize}
  \item \textbf{Marketing -- CMO:} Tiene como objetivo incrementar la visibilidad de \textit{BuroSmart} y captar nuevos clientes mediante campañas de marketing y mejora de la experiencia del usuario
  \item \textbf{Calidad -- CQO:} Tiene como objetivo asegurar que los procesos y productos de \textit{BuroSmart} cumplen con los estándares de calidad y normativas
  \item \textbf{Control financiero y operativo -- COO:} Tiene como objetivo garantizar la estabilidad financiera y la eficiencia operativa de la empresa
  \item \textbf{Control de riesgos y cumplimiento -- CTO:} Tiene como evaluar y mitigar los riesgos empresariales, especialmente en seguridad y cumplimiento normativo
  \item \textbf{Investigación y desarrollo -- CEO:} Tiene como evaluar e impulsar la innovación en la plataforma de \textit{BuroSmart} mediante el desarrollo de nuevas funcionalidades, mejora del rendimiento y adopción de tecnologías emergentes
legales.
\end{itemize}


