\section{Plan de negocio}
En esta sección se presenta el análisis detallado que sustenta la viabilidad y el desarrollo operativo y financiero de BuroSmart. Incluye la organización interna de la empresa, la estructura de costes, las fuentes de financiación, la segmentación de clientes y la política de precios, así como una proyección financiera a medio plazo.

El objetivo es proporcionar un marco claro y fundamentado que permita entender cómo BuroSmart alcanzará sus metas comerciales, garantizará su sostenibilidad y aprovechará las oportunidades de mercado identificadas, respaldando la toma de decisiones estratégicas y facilitando la comunicación con inversores y socios potenciales.

\subsection{Organización de la empresa}
La empresa cuenta con una estructura matricial, la cual ofrece una gran flexibilidad y colaboración entre los distintos departamentos, y permite enfocarse en varias funciones a la vez. Es cierto que este tipo de estructura genera un doble reporte en ciertas situaciones, y puede generar conflictos entre gerentes, pero también permite a cada equipo centrarse en su función sin perder su especialización.

La estructura organizativa se divide por centros de responsabilidad, con un responsable por centro. La \figref{organigrama} muestra el organigrama de la empresa.

\drawiosvgfigure{organigrama}{Organigrama de la empresa}


\subsubsection{Centros de responsabilidad}
Los centros de responsabilidad, con sus respectivos responsables son los siguientes:
\begin{itemize}
  \item \textbf{Marketing -- CMO:} Tiene como objetivo incrementar la visibilidad de \textit{BuroSmart} y captar nuevos clientes mediante campañas de marketing y mejora de la experiencia del usuario
  \item \textbf{Calidad -- CQO:} Tiene como objetivo asegurar que los procesos y productos de \textit{BuroSmart} cumplen con los estándares de calidad y normativas
  \item \textbf{Control financiero y operativo -- COO:} Tiene como objetivo garantizar la estabilidad financiera y la eficiencia operativa de la empresa
  \item \textbf{Control de riesgos y cumplimiento -- CTO:} Tiene como evaluar y mitigar los riesgos empresariales, especialmente en seguridad y cumplimiento normativo
  \item \textbf{Investigación y desarrollo -- CEO:} Tiene como evaluar e impulsar la innovación en la plataforma de \textit{BuroSmart} mediante el desarrollo de nuevas funcionalidades, mejora del rendimiento y adopción de tecnologías emergentes
legales.
\end{itemize}


\subsection{Estructura de costes}

Para elaborar una estructura de costes realista y adaptada a la región de Madrid (España), se han analizado fuentes de ofertas de proveedores tecnológicos relevantes como Glassdoor, Indeed e InfoJobs.

\subsubsection{Costes de Personal}

El equipo inicial de BuroSmart estará compuesto por:

\begin{itemize}
  \item \textbf{Desarrollador de IA y Backend (2 personas):} Salario medio bruto anual en Madrid: 40.000€ por persona. Basado en ofertas de empleo IT en portales como InfoJobs y Glassdoor, además de informes especializados en salarios de perfiles de IA.
  \item \textbf{Desarrollador Frontend (1 persona):} Salario medio bruto anual: 35.000€.
  \item \textbf{Responsable de Calidad y Cumplimiento (1 persona):} Salario medio anual: 38.000€. Este perfil es clave para gestionar la IGPD y compliance legal.
  \item \textbf{Marketing y Ventas (1 persona):} Salario medio anual: 30.000€. Incluye tareas de captación, SEO, SEM basado en estudios de la Asociación Española de Marketing Digital.
  \item \textbf{Soporte y Atención al Cliente (1 persona):} Salario medio anual: 24.000€.
\end{itemize}

\textbf{Total coste anual de personal (6 personas):}  
\[
(2 \times 40.000) + 35.000 + 38.000 + 30.000 + 24.000 = 207.000\,€
\]

Se ha aplicado un 30\% adicional para cargas sociales y beneficios sociales (seguridad social, vacaciones, etc.), lo que equivale a:  
\[
207.000\,€ \times 1.3 = \mathbf{269.100\,€ \text{ anuales}}
\]

\subsubsection{Costes de Hardware e Infraestructura}

BuroSmart se basará en una infraestructura escalable en la nube, fundamental para alojar la plataforma SaaS y ejecutar modelos de lenguaje:

\begin{itemize}
  \item \textbf{Servidores en la nube (Azure/Microsoft):}  
  El coste estimado para un entorno de producción con balanceo de carga y alta disponibilidad es de aproximadamente 1.200€ al mes, que incluye escalado automático, bases de datos gestionadas y seguridad.  
  \textbf{Total anual estimado:} 14.400€
  
  \item \textbf{Licencias y APIs de IA (OpenAI, etc.):}  
  Actualmente OpenAI ofrece servicios de pago según uso, con un coste medio estimado en 2.000€ mensuales para esta escala de proyecto al inicio (asumiendo una carga media mensual).  
  \textbf{Total anual estimado:} 24.000€
  
  \item \textbf{Herramientas y software de desarrollo:}  
  Incluye licencias de software, plataformas de gestión y testing: alrededor de 5.000€ al año.
\end{itemize}

\subsubsection{Costes de Servicios}

\begin{itemize}
  \item \textbf{Marketing digital y eventos:}  
  Campañas mensuales en redes y Google Ads, eventos tecnológicos y colaboraciones con instituciones: 1.000€ al mes (12.000€/año).
  
  \item \textbf{Asesoría legal y cumplimiento:}  
  Servicio especializado para IGPD y elaboración de políticas: unos 10.000€ anuales.
  
  \item \textbf{Costes administrativos y otros:}  
  Servicios de contabilidad, seguros, alquiler de oficina (si se aplica) y gastos generales: 18.000€ anuales.
\end{itemize}

\subsubsection{Resumen de Costes Anuales}

\begin{table}[H]
\centering
\begin{tabular}{|l|r|}
\hline
\textbf{Concepto} & \textbf{Coste Anual (€)} \\
\hline
Personal (6 empleados + cargas) & 269.100 \\
Infraestructura IA y servidores & 14.400 \\
Licencias y APIs IA & 24.000 \\
Herramientas de desarrollo & 5.000 \\
Marketing y eventos & 12.000 \\
Asesoría legal \& compliance & 10.000 \\
Costes administrativos & 18.000 \\
\hline
\textbf{Total Anual de Costes} & \textbf{352.500} \\
\hline
\end{tabular}
\caption{Resumen de costes anuales}
\end{table}

\subsection{Fuentes y Estrategia de Financiamiento Inicial (Funding)}

Para asegurar la viabilidad y expansión los primeros 1-2 años, se prevé obtener financiación que cubra aproximadamente 1,2 millones de euros con el siguiente enfoque:

\subsubsection{Subvenciones y Ayudas Públicas}

\begin{itemize}
  \item \textbf{Plan de Recuperación, Transformación y Resiliencia de la UE – Fondo Next Generation:}  
  España dispone de fondos específicos para transformación digital y desarrollo de IA en el sector público y privado, gestionados por el Ministerio de Asuntos Económicos y Transformación Digital.  
  Se espera obtener al menos \textbf{300.000€ en ayudas y subvenciones públicas} para proyectos de digitalización e innovación tecnológica.
  
  \item \textbf{Ayudas de la Comunidad de Madrid:}  
  La Comunidad dispone de convocatorias locales para startups tecnológicas con ayudas aproximadas de 50.000€ a 100.000€ para proyectos innovadores y de I+D.
\end{itemize}

\subsubsection{Inversores Privados y Venture Capital}

\begin{itemize}
  \item \textbf{Business Angels e Inversores especializados en SaaS/IA:}  
  Firmas y grupos de inversores como \textit{Kibo Ventures, Samaipata Ventures} o \textit{Nauta Capital} interesados en soluciones SaaS para administraciones públicas.  
  La inversión privada estimada en la primera ronda seed es de \textbf{500.000€} a cambio de un 15-20\% de equity.
  
  \item \textbf{Motivación para invertir:}  
  \begin{itemize}
    \item Acceso a un mercado institucional con alto gasto y poca tecnología accesible.  
    \item Potencial escalabilidad regional e internacional.  
    \item Equipo con experiencia y producto diferencial por IA avanzada y cumplimiento normativo.  
    \item Contribución a ODS, atractivo para fondos ESG.
  \end{itemize}
\end{itemize}

\subsubsection{Préstamo Bancario}

Se busca un préstamo bancario de \textbf{100.000€} con interés competitivo (estimado del 5\% anual) para mejorar el cashflow en los primeros meses.  
Requisitos usuales: garantías personales, plan de negocio sólido y plan de devolución mediante ingresos recurrentes.

\subsection{Clientes y Política de Precios}

\subsubsection{Segmento de Clientes}

\begin{itemize}
  \item \textbf{Administración Pública Autónoma y Local:} Enfocado en Comunidad de Madrid y municipios con >50.000 habitantes, usando proyectos piloto y contratos menores.
  \item \textbf{Empresas Privadas Grandes:} Bancos, energéticas, aseguradoras que requieren digitalización y cumplimiento normativo.
\end{itemize}

\subsubsection{Modelo de Ingresos y Tarifas}

Modelo basado en suscripción SaaS y servicios profesionales:

\begin{itemize}
  \item \textbf{Suscripción SaaS (plataforma + IA):}  
  Tarifas mensuales según usuarios y volumen de trámites:  
  - Gobiernos regionales/locales: 3.000€ a 10.000€/mes.  
  - Empresas grandes: 5.000€ a 20.000€/mes.  
  Incluye mantenimiento, actualizaciones y soporte.
  
  \item \textbf{Servicios de consultoría y personalización:}  
  Desde 15.000€ por proyecto o retención anual.
\end{itemize}

\paragraph{Ejemplo de ingresos simulados:}
\begin{itemize}
  \item Comunidad Madrid: 8.000€/mes
  \item Ayuntamiento Madrid: 5.000€/mes
  \item 2 empresas privadas grandes: 12.000€/mes cada una (total 24.000€/mes)
\end{itemize}

\textbf{Ingresos mensuales estimados (año 1):} 37.000€  
\textbf{Ingresos anuales (año 1):} 444.000€

Anticipando crecimiento de cartera del 30\% anual.

\subsection{Proyección Financiera a 5 años}

\begin{table}[H]
\centering
\begin{tabular}{|c|r|r|r|}
\hline
\textbf{Año} & \textbf{Ingresos (€)} & \textbf{Costes (€)} & \textbf{Beneficio Neto (€)} \\
\hline
1 & 444.000 & 352.500 & 91.500 \\
2 & 577.200 & 410.000* & 167.200 \\
3 & 750.360 & 455.000* & 295.360 \\
4 & 975.468 & 505.000* & 470.468 \\
5 & 1.268.108 & 555.000* & 713.108 \\
\hline
\multicolumn{4}{l}{\footnotesize *Incremento estimado en costes por crecimiento de personal y servicios.}
\end{tabular}
\caption{Proyección financiera a 5 años}
\end{table}

El aumento progresivo de costes se debe a:

\begin{itemize}
  \item \textbf{Crecimiento del equipo:} incorporación gradual de personal técnico, comercial y soporte para sostener la expansión.
  \item \textbf{Mayor inversión tecnológica:} escalado en infraestructura cloud, uso creciente de APIs de IA y licencias.
  \item \textbf{Incremento en marketing y ventas:} campañas y presencia en eventos para captar más clientes.
  \item \textbf{Gastos administrativos y legales:} aumento en cumplimiento normativo, asesoría y gestión con el crecimiento.
\end{itemize}
